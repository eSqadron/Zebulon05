\chapter{Sekcja Elektroniczna}
Projekt elektroniczny na potrzeby tego projektu został ograniczony do minimum i był częściowo wzorowany na sekcji elektronciznej projektu Zebulon 2.0. Zasilanie składa się z akumulatora typu LiPo i minimum dwóch przetwornic. Jako mózg urządzenia zastosowany został minikomputer Raspberry Pi 4B i to do jego zasilenia potrzebna jest jedna z przetwornic. Zgodnie z dokumentacją Raspberry zasilanie jest ze źródła o napięciu $5V$ i prądzie przynajmniej $3A$.\cite{RPI_power_sup} Dlatego została zainstalowana przetwornica TODO.\\

Druga przetwornica ma za zadanie zasilić serwomechanizmy. Serwomechanizmy wybrane do tego projektu to Feetech FT5715M (sztuk 3) i PowerHD LF-20MG (sztuk 6). W czasie zakupu serwa te wypadały najlepiej spośród wszystkich dostępnych pod kątem prędkości ruchu do ceny. Serwa firmy Feetech są zasilanie napięciem z zakresu $4.8$ do $6V$ \cite{feetech_docs} a Power HD napięciem $4.8$ do $6.6V$ \cite{powerhd_docs}. Dlatego jako wspólne napięcie zasilania ustalona została wartość $6V$. Jako że zwykle serwo pobiera do około $1A$ prądu to do ich zasilenia potrzeba przetwornicę o wyjściu $6V$ i minimum $9A$ \cite{Servo_power_sup}. Natomiast należy pamiętać że dobieranie przetwornicy "na styk" przy czymś takim jak zasilanie serwomechanizmów może spowodować później problemy przy większych obciążeniach. Dlatego, aby uwzględnić pewien zapas prądowy, wybrana została przetwornica TODO

%https://botland.com.pl/serwa-typu-standard/9182-serwo-feetech-ft5715m-standard-5904422312756.html
%https://botland.com.pl/serwa-typu-standard/3576-serwo-powerhd-lf-20mg-standard-6939670200387.html