\documentclass[a4paper,13pt]{article}
\usepackage[MeX]{polski}
\usepackage[hidelinks]{hyperref}
\usepackage{color}
\usepackage{mathtools}
\usepackage{float}
\usepackage[OT4]{fontenc}
\usepackage{listings}
\usepackage[margin=1in]{geometry}

\frenchspacing




\lstset{
	language=C++,
	breaklines=true,
	postbreak=\mbox{\textcolor{red}{$\hookrightarrow$}\space},
	literate={ą}{{\,a}}1 {ę}{{\,e}}1 {ó}{{\'o}}1 {ż}{{\*z}}1 {ź}{{\'z}}1 {ł}{{\'l}}1 {ś}{{\'s}}1 {ć}{{\'c}}1 {ń}{{\'n}}1
}


\title{Projekt robota trójnożnego}
\author{Jakub Mazur }
\date{\today}


\begin{document}

%\begin{lstlisting}
%ą ę ż ł
%\end{lstlisting}
% TODO - ogarnąć encoding


\maketitle

\hypersetup{
	linktocpage=true,
    colorlinks=true,
    urlcolor=red,
    linktoc=all,
    linkcolor=blue,
}
\tableofcontents

\section{Noga robota}
Noga ma 3 stopnie swobody. Wszystkie są typu obrotowego, przy czym dwie obracają się dookoła osi poziomej, a jedna dookoła osi pionowej. Są to te same osi obrotu co w przypadku ramienia robotycznego typu antromorficznego.
\subsection{Model matematyczny}
\subsubsection{Forward kinematics}

\begin{equation} \label{eq1}
\begin{split}
a_{temp} &= a_2 \cos{\alpha_1} + a_3 \cos{\left(\alpha_2 - \alpha_1\right)} + a_1\\
Y &= a_{temp} \cdot \sin{\alpha_0}\\
X &= a_{temp} \cdot \cos{\alpha_0}\\
Z &= h_0 - h_1 + a_2 \sin{\alpha_1} - a_3 \sin{\left(\alpha_2 - \alpha_1\right)}
\end{split}
\end{equation}

\subsubsection{Forward kinematics - denavit hartenberg \cite{DH_AA_article}}

\[
\begin{array}{c|cccc}
\textrm{Joint } i & \theta_i & \alpha_i & r_i & d_i \\
1 
\end{array}
\]
Gdzie: \\
$\theta_i$ - angle from $x_{n-1}$ to $x_n$ around $z_{n-1}$\\
$\alpha_i$ - angle from $z_{n-1}$ to $z_n$ around $x_n$\\
$r_i$ - distance between the origin of the $n-1$ frame and the origin of the $n$ frame along the $x_n$ direction.\\ 
$d_i$ - distance from $x_{n-1}$ to $x_n$ along the $z_{n-1}$ direction\\
\subsubsection{Invert kinematics}
Odwrotną kinematykę można obliczyć poprzez rozwiązanie równań kinematyki prostej.

Najprościej jest wyprowadzić wzór na $\alpha_0$, można to zrobić łącząc wzór na $X$ i $Y$:
\begin{equation}
\begin{split}
\begin{cases}
Y = a_{temp} \cdot \sin{\alpha_0}\\
X = a_{temp} \cdot \cos{\alpha_0}
\end{cases}
\end{split}
\end{equation}

\begin{equation}
\begin{split}
\begin{cases}
Y = a_{temp} \cdot \sin{\alpha_0}\\
a_{temp} = \frac{X}{\cos{\alpha_0}}
\end{cases}
\end{split}
\end{equation}


\begin{equation}
Y = \frac{X}{\cos{\alpha_0}} \cdot \sin{\alpha_0}\\
\end{equation}

\begin{equation}
\alpha_0 = \arctan{\frac{Y}{X}}
\end{equation}


\begin{thebibliography}{9}
\bibitem{DH_AA_article}
\href{https://automaticaddison.com/how-to-find-denavit-hartenberg-parameter-tables/}{How to Find Denavit-Hartenberg Parameter Tables, blogpost by Automatic Addison}

\bibitem{}
\href{https://www.diva-portal.org/smash/get/diva2:1462059/FULLTEXT01.pdf}{Alexander Wallen Kiessling, Niclas Maatta (2020) Anthropomorphic Robot Arm}

\end{thebibliography}
\end{document}